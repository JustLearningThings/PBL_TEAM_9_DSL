\chapter*{Abstract}

The project entitled "Domain Specific Language for Unit Testing" was developed by the students: Smocvin Denis, Chiper Andreea, Guzun Cezar, Telug Anatolie, Luchianov Vladimir from Technical University of Moldova. This project consists of  chapters: introduction, domain analysis, language overview, grammar presentation, implementation, conclusions and references.

\textbf{Keywords: }  domain specific language, bugs, developers, module, parser, grammar, suite.
% Put your keywords of choice.

The purpose of this report is to present the development of a project based on the creating of a domain specific language for unit testing. 
Unit testing is the process of testing individual pieces of code. Even while it is incredibly useful, it has a few drawbacks that can be solved by a domain-specific language. 
The domain specific language for unit testing is designed to improve the ease and reliability of unit testing. By allowing developers to write code in a language that is more intuitive and consistent, they can save time, resources, and energy when creating and executing unit tests. 
The need of a domain specific language in unit testing will be discussed, as well as how to design one so that users can get the most out of it. The grammar and parser of the unit testing will also be explained.
The tools used were ANTLR and TextX to define the language structure and  to create the parser for the grammar
